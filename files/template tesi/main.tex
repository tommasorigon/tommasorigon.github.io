\documentclass[12pt, a4paper, twoside, openright]{book} 
\usepackage[top=3cm, bottom=3cm, left=3cm, right=3cm]{geometry}

\usepackage[italian]{babel} 
\usepackage[T1]{fontenc}
\usepackage[utf8x]{inputenc} % Serve per le lettere accentate
\usepackage{mathtools} % per valore assoluto e norma
\usepackage{graphicx} % Comandi aggiuntivi per la gestione delle immagini
\usepackage{url} % per scrivere gli indirizzi Internet
\usepackage{booktabs,subcaption} % Tabelle
\usepackage{amsthm,amsmath,amssymb,mathrsfs,dsfont}
\usepackage{rotating}
\usepackage[font=itshape,noorphans]{quoting}
\usepackage{bm,bbm}
\usepackage{microtype}
\microtypesetup{protrusion=true,expansion=true}

\usepackage{hyperref}
\usepackage[dvipsnames]{xcolor}

\usepackage{tikz}
\usetikzlibrary{fit,calc,positioning,shapes,shadows,arrows}
\tikzset{font={\fontsize{10pt}{12}\selectfont}}

\usepackage{scrextend}
\usepackage[ruled,norelsize,vlined]{algorithm2e}

\usepackage[round]{natbib}
\usepackage[nottoc]{tocbibind}
\addto\captionsenglish{\renewcommand{\bibname}{Bibliografia}}

\usepackage[small]{titlesec}

\usepackage{caption}
\captionsetup{labelfont={bf},font=small}
\captionsetup[sub]{font=small,labelfont={bf}}

\usepackage{setspace}
\onehalfspacing
\linespread{1.3}

%These commands assume a very low tolerance for spacing between paragraphs
\usepackage[all]{nowidow} % Somewhat extreme, but it avoids widow rows
\pretolerance=0
\lineskip=0pt \lineskiplimit=0pt %
\tolerance=2000 \hyphenpenalty=300 \exhyphenpenalty=300%
\doublehyphendemerits=100000%
\finalhyphendemerits=\doublehyphendemerits
 

%\usepackage[osf]{mathpazo} % add possibly `sc` and `osf` options
%\usepackage[euler-digits]{eulervm}
\usepackage[scaled=0.85]{beramono}
\DeclareSymbolFont{greekletters}{OML}{cmr}{m}{it}
\DeclareMathSymbol{\varrho}{\mathalpha}{greekletters}{"25}
\DeclareMathSymbol{\varsigma}{\mathalpha}{greekletters}{"26}

% COlors
\definecolor{webgreen}{rgb}{0,0.5,0}
\definecolor{webbrown}{rgb}{0.6,0,0}
\definecolor{structural}{RGB}{0,0, 130}   % Structural


%%%%%%%%%%%%%%%%%%%%%%%%%%%%%
%Opzioni pacchetto Fancyhdr
%%%%%%%%%%%%%%%%%%%%%%%%%%%%%

\usepackage{fancyhdr}

\pagestyle{fancy}
\setlength{\headheight}{15pt}
% i comandi seguenti impediscono la scrittura in maiuscolo
% dei nomi dei capitoli e dei paragrafi nelle intestazioni
\renewcommand{\chaptermark}[1]{\markboth{#1}{}}
\renewcommand{\sectionmark}[1]{\markright{\thesection\ #1}}
\fancyhf{} % rimuove l'attuale contenuto dell'intestazione
\fancyhead[LE,RO]{\thepage \sffamily}
\fancyhead[LO]{Capitolo \thechapter. \textit{\leftmark}}
\fancyhead[RE]{Capitolo \thechapter. \textit{\leftmark}}
\renewcommand{\headrulewidth}{0.5pt}
\renewcommand{\footrulewidth}{0pt}

% Acknloedgments settings
\newenvironment{acknowledgements}% 
{\cleardoublepage\null\vfill\begin{center} %
\bfseries Ringraziamenti\end{center}} %
{\vfill\null}

%*********************************************************************************
% Nuovi ambienti: definizioni, teoremi etc. etc.
%*********************************************************************************
% definizioni (serve il pacchetto amsthm)
\newtheoremstyle{classicdef}% Nome
{12pt}% Spazio che precede l’enunciato
{12pt}% Spazio che segue l’enunciato
{}% Stile del font dell’enunciato
{}% Rientro (se vuoto, non c’è rientro,
% \parindent = rientro dei capoversi)
{\scshape}% Stile del font dell’intestazione
{:}% Punteggiatura che segue l’intestazione
{.5em}% Spazio che segue l’intestazione:
% " " = normale spazio inter-parola;
% \newline = a capo
{}% Specifica l’intestazione dell’enunciato
% (normalmente viene lasciata vuota)

\theoremstyle{definition}
\newtheorem{assumption}{Assunzione}
\numberwithin{assumption}{chapter}

\theoremstyle{definition}
\newtheorem{definition}{Definizione}
\numberwithin{definition}{chapter}

\theoremstyle{remark}
\newtheorem{example}{Esempio}
\numberwithin{example}{chapter}

\theoremstyle{remark}
\newtheorem{remark}{Nota}
\numberwithin{remark}{chapter}


% teoremi (serve il pacchetto amsthm)
\newtheoremstyle{classicthm}% Nome
{12pt}% Spazio che precede l’enunciato
{12pt}% Spazio che segue l’enunciato
{\itshape}% Stile del font dell’enunciato
{}% Rientro (se vuoto, non c’è rientro,
% \parindent = rientro dei capoversi)
{\scshape}% Stile del font dell’intestazione
{:}% Punteggiatura che segue l’intestazione
{.5em}% Spazio che segue l’intestazione:
% " " = normale spazio inter-parola;
% \newline = a capo
{}% Specifica l’intestazione dell’enunciato
% (normalmente viene lasciata vuota)

\theoremstyle{plain}
\newtheorem{theorem}{Teorema}
\numberwithin{theorem}{chapter}

\newtheorem{corollary}{Corollario}
\numberwithin{corollary}{chapter}

\newtheorem{lemma}{Lemma}
\numberwithin{lemma}{chapter}

\newtheorem{proposition}{Proposizione}
\numberwithin{proposition}{chapter}


%*********************************************************************************
% Impostazioni di hyperref
%*********************************************************************************
\hypersetup{%
    % hyperfootnotes=false,pdfpagelabels,
    %draft,	% = elimina tutti i link (utile per stampe in bianco e nero)
    colorlinks=true, linktocpage=true, pdfstartpage=1, 
    % decommenta la riga seguente per avere link in nero (per esempio per la stampa in bianco e nero)
    % colorlinks=false, linktocpage=false, pdfborder={0 0 0}, pdfstartpage=1, pdfstartview=FitV,%
    breaklinks=true, pdfpagemode=UseNone, pageanchor=true, pdfpagemode=UseOutlines,%
    plainpages=false, bookmarksnumbered, bookmarksopen=true, bookmarksopenlevel=1,%
    hypertexnames=true, pdfhighlight=/O,%nesting=true,%frenchlinks,%
    urlcolor=webbrown, linkcolor=Maroon, citecolor=structural, %pagecolor=RoyalBlue,%
    %urlcolor=Black, linkcolor=Black, citecolor=Black, %pagecolor=Black,%
}

%%%%%%%%%%%%%%%%%%%%%%%%%%%%%%%%%%%
% Definizione di alcuni parametri
%%%%%%%%%%%%%%%%%%%%%%%%%%%%%%%%%%%

\graphicspath{{img/}} % Directory delle immagini

%%%%%%%%%%%%%%%%%%%%%%%%%%%%%%%%%%
% INIZIO DEL DOCUMENTO
%%%%%%%%%%%%%%%%%%%%%%%%%%%%%%%%%%

\begin{document}

\frontmatter

\pagestyle{empty}



\pagestyle{empty} 

\begin{titlepage}
  
 \begin{center}
 {\large  

 \hfill

 \vfill
 {
 {\Large \textsc{Università degli studi di Milano--Bicocca}}\\
 {\textsc{Scuola di Economia e Statistica}}\\
 \vfill
	\textsc{Corso di Laurea  in} \\
	\textsc{Scienze Statistiche ed Economiche} \\
	\vfill
	\includegraphics[width=4cm]{front}
 \vfill
 
 {\Huge\color{Maroon}\textsc{Titolo elaborato}}\\
 }
}
\end{center}

\vfill
{
\large
\begin{flushleft}
\textsc{Relatore}: Dott. Tommaso Rigon \\
\end{flushleft}

\vfill
\begin{flushright}
\textsc{Tesi di laurea di}:\\
 Nome Cognome\\
\textsc{Matricola N. 123456}
\end{flushright}

\vfill
\begin{center}
\textsc{Anno Accademico 20XX/20YY}
\end{center}

%\vfill
%\begin{flushright}
%\textsc{\today}
%\end{flushright}
}
\end{titlepage}


% Lista dei contenuti

\tableofcontents


\begin{acknowledgements}
Inserire qui gli eventuali ringraziamenti, altrimenti eliminare 
\end{acknowledgements}

\cleardoublepage

\mainmatter

\pagestyle{fancy}

\chapter{Lorem Ipsum}
Lorem ipsum dolor sit amet, consectetur adipiscing elit. Morbi vehicula rhoncus venenatis. Sed eu mauris ut risus tempor faucibus. Donec tincidunt congue faucibus. Quisque cursus egestas eleifend. Vivamus at mi vel erat suscipit ullamcorper. Proin mauris sem, rutrum sit amet sollicitudin nec, varius id tellus. Vestibulum porttitor ultricies congue.

\section{Prima sezione}
Lorem ipsum dolor sit amet, consectetur adipiscing elit. Morbi vehicula rhoncus venenatis. Sed eu mauris ut risus tempor faucibus. Donec tincidunt congue faucibus. Quisque cursus egestas eleifend. Vivamus at mi vel erat suscipit ullamcorper. Proin mauris sem, rutrum sit amet sollicitudin nec, varius id tellus. Vestibulum porttitor ultricies congue.

\subsection{Prima sottosezione}
Lorem ipsum dolor sit amet, consectetur adipiscing elit. Morbi vehicula rhoncus venenatis. Sed eu mauris ut risus tempor faucibus. Donec tincidunt congue faucibus. Quisque cursus egestas eleifend. Vivamus at mi vel erat suscipit ullamcorper. Proin mauris sem, rutrum sit amet sollicitudin nec, varius id tellus. Vestibulum porttitor ultricies congue.
\subsubsection{Sotto-sotto-sezione}
Lorem ipsum dolor sit amet, consectetur adipiscing elit. Morbi vehicula rhoncus venenatis. Sed eu mauris ut risus tempor faucibus. Donec tincidunt congue faucibus. Quisque cursus egestas eleifend. Vivamus at mi vel erat suscipit ullamcorper. Proin mauris sem, rutrum sit amet sollicitudin nec, varius id tellus. Vestibulum porttitor ultricies congue.
\par 
Lorem ipsum dolor sit amet, consectetur adipiscing elit. Morbi vehicula rhoncus venenatis. Sed eu mauris ut risus tempor faucibus. Donec tincidunt congue faucibus. Quisque cursus egestas eleifend. Vivamus at mi vel erat suscipit ullamcorper. Proin mauris sem, rutrum sit amet sollicitudin nec, varius id tellus. Vestibulum porttitor ultricies congue. Lorem ipsum dolor sit amet, consectetur adipiscing elit. Morbi vehicula rhoncus venenatis. Sed eu mauris ut risus tempor faucibus. Donec tincidunt congue faucibus. Quisque cursus egestas eleifend. Vivamus at mi vel erat suscipit ullamcorper. Proin mauris sem, rutrum sit amet sollicitudin nec, varius id tellus. Vestibulum porttitor ultricies congue.


\chapter{Esempi}


\begin{theorem}[Teorema di Pitagora] Dato un triangolo rettangolo di lati $a,b$ e $c$, dove $c$ è l'ipotenusa e $a$ e $b$ sono i cateti, allora

$$
a^2 + b^2 = c^2.
$$

\end{theorem}

\subsubsection{Esempio formula matematica}
\begin{equation}
3^2+4^2=5^2
\end{equation}


\subsubsection{Esempio matrice}
\begin{equation}
j(\tau,\zeta) = \begin{bmatrix}
j_{\zeta \zeta}(\tau,\zeta) & j_{\zeta \tau}(\tau,\zeta) \\
j_{\tau \zeta}(\tau,\zeta) &  j_{\tau \tau}(\tau,\zeta)    
\end{bmatrix}
\end{equation}


\subsubsection{Esempio citazione}
Si veda \citet{pacesalvan}


\subsubsection{Esempio figura (con etichetta)}
Si veda la \figurename~\ref{logo}
\begin{figure}[htbp]
\begin{center}
\includegraphics[width=0.30\textwidth]{front}
\caption{Il logo Unimib \label{logo}}
\end{center}
\end{figure}

\subsubsection{Esempio tabella (con etichetta)}
Si veda la \tablename~\ref{tab}
\begin{table}[htbp]
\caption{Sommario di R  \label{tab}}
\centering
\begin{tabular}{lcccr}
  \toprule
 & Estimate & Std. Error & t value & Pr($>$  |t|) \\ 
  \midrule
(Intercept) & 1.2771 & 0.4844 & 2.64 & 0.0113 \\ 
  speed & 0.3224 & 0.0298 & 10.83 & 0.0000 \\ 
   \bottomrule
\end{tabular}
\end{table}


%\backmatter

\cleardoublepage
%\addcontentsline{toc}{chapter}{References}

\fancyhf{} % rimuove l'attuale contenuto dell'intestazione
\fancyhead[LE,RO]{\thepage \sffamily}
\fancyhead[LO,RE]{\textit{Bibliografia}}

\bibliographystyle{biblio_style.bst}
\bibliography{biblio} %Bibliography file

\end{document}